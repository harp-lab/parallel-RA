%
% The first command in your LaTeX source must be the \documentclass command.
\documentclass[sigconf,review, anonymous]{acmart}


%
% defining the \BibTeX command - from Oren Patashnik's original BibTeX documentation.
\def\BibTeX{{\rm B\kern-.05em{\sc i\kern-.025em b}\kern-.08emT\kern-.1667em\lower.7ex\hbox{E}\kern-.125emX}}
    
% Rights management information. 
% This information is sent to you when you complete the rights form.
% These commands have SAMPLE values in them; it is your responsibility as an author to replace
% the commands and values with those provided to you when you complete the rights form.
%
% These commands are for a PROCEEDINGS abstract or paper.
\copyrightyear{2018}
\acmYear{2018}
\setcopyright{acmlicensed}
\acmConference[Woodstock '18]{Woodstock '18: ACM Symposium on Neural Gaze Detection}{June 03--05, 2018}{Woodstock, NY}
\acmBooktitle{Woodstock '18: ACM Symposium on Neural Gaze Detection, June 03--05, 2018, Woodstock, NY}
\acmPrice{15.00}
\acmDOI{10.1145/1122445.1122456}
\acmISBN{978-1-4503-9999-9/18/06}

%
% These commands are for a JOURNAL article.
%\setcopyright{acmcopyright}
%\acmJournal{TOG}
%\acmYear{2018}\acmVolume{37}\acmNumber{4}\acmArticle{111}\acmMonth{8}
%\acmDOI{10.1145/1122445.1122456}

%
% Submission ID. 
% Use this when submitting an article to a sponsored event. You'll receive a unique submission ID from the organizers
% of the event, and this ID should be used as the parameter to this command.
%\acmSubmissionID{123-A56-BU3}

%
% The majority of ACM publications use numbered citations and references. If you are preparing content for an event
% sponsored by ACM SIGGRAPH, you must use the "author year" style of citations and references. Uncommenting
% the next command will enable that style.
%\citestyle{acmauthoryear}
\usepackage{graphicx}
\usepackage{tikz}
\usepackage{caption}
\usepackage{subcaption}


%% Tikz
\usetikzlibrary{positioning}
\tikzset{main node/.style={circle,fill=gray!20,draw,minimum size=0.45cm,inner sep=0pt}, }
\tikzset{invisible/.style={circle,fill=white!20,draw,minimum size=0.0cm,inner sep=0pt}, }


%
% end of the preamble, start of the body of the document source.
\begin{document}

%
% The "title" command has an optional parameter, allowing the author to define a "short title" to be used in page headers.
\title{Distributed Relational Algebra at Scale}

%
% The "author" command and its associated commands are used to define the authors and their affiliations.
% Of note is the shared affiliation of the first two authors, and the "authornote" and "authornotemark" commands
% used to denote shared contribution to the research.
\author{Sidharth Kumar}
\authornote{Both authors contributed equally to this research.}
\email{sid14@uab.edu}
\orcid{}
\author{Thomas Gilray}
\authornotemark[1]
\email{gilray@uab.edu}
\affiliation{%
  \institution{University of Alabama, Birmingham}
  \streetaddress{}
  \city{Birmingham}
  \state{Alabama}
  \postcode{35233}
}

\newenvironment{tightenumerate}{
	\edef\backupindent{\the\parindent}
	\begin{enumerate}
		\setlength{\itemsep}{1pt}
		\setlength{\parskip}{0pt}
		\setlength{\parsep}{0pt}
		\setlength{\parindent}{\backupindent}
	}{\end{enumerate}
}
\newenvironment{tightitemize}{
	\edef\backupindent{\the\parindent}
	\begin{itemize}
		\setlength{\itemsep}{1pt}
		\setlength{\parskip}{0pt}
		\setlength{\parsep}{0pt}
		\setlength{\parindent}{\backupindent}
	}{\end{itemize}
}

%\author{Julius P. Kumquat}
%\affiliation{\institution{The Kumquat Consortium}}
%\email{jpkumquat@consortium.net}

%
% By default, the full list of authors will be used in the page headers. Often, this list is too long, and will overlap
% other information printed in the page headers. This command allows the author to define a more concise list
% of authors' names for this purpose.
\renewcommand{\shortauthors}{Kumar and Gilray}

%
% The abstract is a short summary of the work to be presented in the article.
\begin{abstract}
%Relational algebra forms a basis of primitive operations useful for applications in graphs, networks, program analysis, deductive databases, and logic. Despite its expressive power, relational algebra has not received the same attention in high-performance computing research as linear algebra, stencil computation, map-reduce and graph analytics.

%In this paper we present a set of efficient algorithms that tackle the problem of distributed, parallel relational algebra and use experiments from applications in graphs, program analysis, and datalog to evaluate our approach.
  Relational algebra forms a basis of primitive operations suitable for applications in graphs and networks, program analysis, deductive databases, and constraint logic programming. Despite its expressive power, relational algebra has not received the same attention in high-performance-computing research as more common primitives like stencil computations, floating-point operations, numerical integration, and sparse linear algebra. Furthermore, specific challenges in addressing representation and communication among distributed portions of a relation have previously thwarted successful scaling of relational algebra applications to supercomputers.
  
In this paper, we present a set of efficient algorithms to effectively parallelize and scale key relational algebra primitives. We introduce a hybrid hash-tree approach to representing distributed relations and permitting efficient communication. Finally, we demonstrate the scalability of our implementation with a fixed-point algorithm computing the transitive closure of a large graph (generating over $276$ billion edges) on $65,\!536$ nodes.
\end{abstract}

%
% The code below is generated by the tool at http://dl.acm.org/ccs.cfm.
% Please copy and paste the code instead of the example below.
%
\begin{CCSXML}
<ccs2012>
 <concept>
  <concept_id>10010520.10010553.10010562</concept_id>
  <concept_desc>Computer systems organization~Embedded systems</concept_desc>
  <concept_significance>500</concept_significance>
 </concept>
 <concept>
  <concept_id>10010520.10010575.10010755</concept_id>
  <concept_desc>Computer systems organization~Redundancy</concept_desc>
  <concept_significance>300</concept_significance>
 </concept>
 <concept>
  <concept_id>10010520.10010553.10010554</concept_id>
  <concept_desc>Computer systems organization~Robotics</concept_desc>
  <concept_significance>100</concept_significance>
 </concept>
 <concept>
  <concept_id>10003033.10003083.10003095</concept_id>
  <concept_desc>Networks~Network reliability</concept_desc>
  <concept_significance>100</concept_significance>
 </concept>
</ccs2012>
\end{CCSXML}

\ccsdesc[500]{Computer systems organization~Embedded systems}
\ccsdesc[300]{Computer systems organization~Redundancy}
\ccsdesc{Computer systems organization~Robotics}
\ccsdesc[100]{Networks~Network reliability}

%
% Keywords. The author(s) should pick words that accurately describe the work being
% presented. Separate the keywords with commas.
\keywords{datasets, neural networks, gaze detection, text tagging}



%
% This command processes the author and affiliation and title information and builds
% the first part of the formatted document.
\maketitle



\section{Introduction}
\label{sec:intro}
%
Implementing application-specific code on supercomputers requires addressing the fundamental underlying primitives of an algorithm in a way that is flexible and scalable. Significant progress has been made on a wide variety of important problems due to a rigorous exploration of common high-performance primitives such as stencil computations, floating-point arithmetic, numerical integration, and sparse linear algebra. 

Relational algebra is a crucial primitive for a wide range of analytic problems in graphs, machine learning, logic programming, program analysis, deductive databases, and formal verification, that has been the subject of great interest in the literature, but has had limited exploration on supercomputers, and at scale. Two central barriers to scaling operations on relations, such as union, selection, projection, and join, have been (a) how to represent distributed relations in a way that is amenable to efficient parallel operations, and (b) how to handle communication to coordinate distinct portions of the distributed workload.

While some recent progress has been made in addressing these issues, (a) in particular, no approach has yet provided a general framework that makes applications using a pipeline of \emph{repeated} operations on relations---for fixed-point interation, supporting applications such as Datalog and program analysis---possible at scale. For such applications to be implemented on distributed, many-core systems, existing algorithms that distribute relations among available cores, perform a single operation, and return in map-reduce fashion are not suitable as repeated operations require efficient granular communication at each step. 

In this paper, we present a hybrid approach to representing relations on networked machines and performing efficient distributed operations on them, building on the current state of the art for single-node parallelism. Interestingly, in addressing the communication issue, we find that MPI's all-to-all communication paradigm suits relational algebra best. Today's supercomputers have specialized, high speed interconnects and data can be transmitted between processes with very low latency. When used with an appropriate configuration, all-to-all communication---known to be the most intensive mode of communication---can scale well.


\paragraph{Contributions} In particular, we make the following specific contributions to the literature:
\begin{tightenumerate}
	\item We present novel hybrid hash-tree based algorithms for distributed relational algebra.
	\item We present a scalable implemetation and experiments for a fixed-point algorithm employing distributed relational algebra: computing the transitive closure of a graph.
	\item We demonstrate scalability of transitive closure up to $65,\!536$ processes, producing a graph with more than 276 billion edges. To the best of our knowledge, this is the largest transitive closure operation discussed in the literature. 
\end{tightenumerate}

We understand our implementation to be the first truly scalable distributed relational algebra that addresses inter-process communication, permitting fixed-point interation, and laying the foundation for solving massive logical inference problems, graph problems, and more, on supercomputers.









\section{Relational Algebra}
\label{sec:ra}
%
Relational algebra (RA) provides a basis of operations on relations (i.e., predicates, or sets of tuples) sufficient to implement a broad range of algorithms for databases and queries, data analysis, machine learning, graph problems, and constraint logic problems \cite{}. Scaling these underlying primitives, and finding an effective strategy for parallel communication to distribute them across multiple nodes, is thus a avenue for scaling and distributing algorithms for high-performance program analyses, deductive databases, among other applications. This section reviews the standard relational operations union, product, intersection, natural join, selection, renaming, and projection, along with their use in implementing two closely related example applications: graph problems and datalog solvers.

The Cartesian product of two finite enumerations $D_0$ and $D_1$ is defined $D_0 \times D_1 = \{ (d_0, d_1) \ |\ \forall d_0 \in D_0, d_1 \in D_1 \}$. A \textit{relation} $R \subseteq D_0 \times D_1$ is some subset of this product that defines a set of associated pairs of elements drawn from the two domains. For example, if R were the relation $(\geq)$ over natural numbers, both domains $D_0$ and $D_1$ would be $\mathbb{N}$ and the relation could be defined $(\geq) = \{ (n_0, n_1) \ |\ n_0, n_1 \in \mathbb{N} \wedge n_0 \geq n_1 \}$. Any relation $R$ can also be viewed as a predicate $P_R$ where $P_R(d_0, \ldots, d_k) \iff (d_0, \ldots, d_k) \in R$, or as a set of tuples, or as a database table.

We make some standard assumptions about relational algebra that differ from those of traditional set operations. Specifically, we assume that all our relations are sets of flat (first-order) tuples of natural numbers with a fixed, homogeneous arity. This means that the relation $(\mathbb{N} \times \mathbb{N}) \times \mathbb{N}$ contains the tuple $(1,2,3)$, and not $((1,2),3)$. It also means that although our approach extends naturally to relations over arbitrary enumerable domains (such as integers, booleans, symbols/strings, lists of integers, etc)---we make the assumption that natural numbers may be used in the place of other enumerable domains when they are needed. Finally, this means that for operations like union or intersection, both relations must by union-compatable by having the same arity and column names.    

... talk about names as indices?


\subsection{Standard RA operations}
\label{sec:ra:tc}
%
...

\paragraph{Cartesian product} The product of two relations $R$ and $S$ is defined: $R \times S = \{ (r_0, \ldots, r_k, s_0, \ldots, s_j) \ |\ (r_0, \ldots, r_k) \in R \wedge (s_0, \ldots, s_j) \in S \}$.


\paragraph{Union} The union of two relations $R$ and $R'$ may only be performed if both relations have the same arity but is otherwise set union: $R \cup R' = \{ (r_0, \ldots, r_k) \ |\ (r_0, \ldots, r_k) \in R \vee (r_0, \ldots, r_k) \in R' \}$.


\paragraph{Intersection} The intersection of two relations $R$ and $R'$ may only be performed if both have $k$ arity but is otherwise set intersection: $R \cap R' = \{ (r_0, \ldots, r_k) \ |\ (r_0, \ldots, r_k) \in R \wedge (r_0, \ldots, r_k) \in R' \}$.


\paragraph{Projection} Projection is a unary operation that removes a column or columns from a relation---and thus any duplicate tuples that result from removing these columns. Projection of a relation $R$ restricts $R$ to a particular set of dimensions ${\alpha_0, \ldots, \alpha_j}$, where $\alpha_0 < \ldots < \alpha_j$, and is written $\Pi_{\alpha_0,\ldots,\alpha_j}(R)$. For each tuple, projection retains only stated columns: $\Pi_{\alpha_0,\ldots,\alpha_j}(R) = \{ (r_{\alpha_0}, \ldots, r_{\alpha_j}) \ |\  (r_0, \ldots, r_k) \in R \}$.


\paragraph{Renaming} Renaming is a unary operation that renames (i.e., reorders) columns. Renaming columns can be defined in several different ways, including renaming all columns at once. We define our renaming operator, $\rho_{\alpha_i / \alpha_j}(R)$, to swap two columns, $\alpha_i$ and $\alpha_j$ where $\alpha_i < \alpha_j$---an operation that can be repeated to rename/reorder as many columns as desired: \newline$\rho_{\alpha_i / \alpha_j}(R) = \{ (\ldots,r_{\alpha_j},\ldots,r_{\alpha_{i}},\ldots) \ |\ (\ldots,r_{\alpha_{i}},\ldots,r_{\alpha_{j}},\ldots) \in R \}$.


\paragraph{Selection} Selection is a unary operation that restricts a relation to tuples where a particular column matches a particular value. As with renaming, a selection operator may alternatively be defined to allow multiple columns to be matched at once, or to allow inequality or other predicates to be used in matching tuples. In our formulation, selection on multiple columns can be accomplished by repeated selection on a single column at a time. Selecting just those tuples from relation R where column $\alpha_i$ matches value $v$ is performed with operator $\sigma_{\alpha_i = v}(R)$ that is defined: \newline$\sigma_{\alpha_i = v}(R) = \{ (r_{\alpha_0}, \ldots, r_{\alpha_k}) \ |\ (r_{\alpha_0}, \ldots, r_{\alpha_i}, \ldots, r_{\alpha_k}) \in R \wedge r_{\alpha_i} = v \}$.

\paragraph{Natural Join} Two relations can also be \textit{joined} into one on a subset of columns they have in common. Join is a particularly important operation that combines two relations into one, where a subset of columns are required to have matching values, and generalizes both intersection and Cartesian product operations.

Consider an example of two tables in a database, one that encodes a system's users' \texttt{emails} (including their username, email address, and whether it's verified) and another that encodes successful \texttt{logins} (including a username, timestamp, and ip address):

\begin{center}
  \textbf{\texttt{emails}} \vspace{0.05cm} \\
  \begin{tabular}{ | c | c | c | }
    \hline
    \textbf{username} & \textbf{email} & \textbf{verified} \\
    \hline
    \texttt{samp} & \texttt{samwow@gmail.com} & \texttt{1} \\ \hline
    \texttt{samp} & \texttt{samp9@uab.edu} & \texttt{0} \\ \hline
    \texttt{karenk} & \texttt{karenk5@uab.edu} & \texttt{1} \\ \hline
  \end{tabular}
  \vspace{0.3cm} \\
  \textbf{\texttt{logins}} \vspace{0.05cm} \\
  \begin{tabular}{ | c | c | c | }
    \hline
    \textbf{\texttt{username}} & \textbf{\texttt{timestamp}} & \textbf{\texttt{address}} \\
    \hline
    \texttt{samp} & \texttt{1554291414} & \texttt{162.103.150.12} \\ \hline
    \texttt{karenk} & \texttt{1554181337} & \texttt{171.31.15.120} \\ \hline
    \texttt{karenk} & \texttt{1554219962} & \texttt{155.28.11.102} \\ \hline
    \texttt{karenk} & \texttt{1554133720} & \texttt{171.31.15.120} \\ \hline
  \end{tabular}
\end{center}

A join operation on these two relations, written $\texttt{users} \bowtie \texttt{logins}$,
yields a single relation with all five columns: username, email, passhash, timestamp, address. For columns the two relations have in common, the natural join only considers pairs of tuples from the two input relations where the values for those columns match, as in an intersection operation; for other columns, the natural join computes all possible combinations of their values as in Cartesian product. If both input relations share all columns in common, a join is simply intersection and if both input relations share no columns in common, a join is simply Cartesian product. For the above tables, the natural join

\begin{center}
  $\textbf{\texttt{emails}} \bowtie \textbf{\texttt{logins}}$ \vspace{0.05cm} \\
  \begin{tabular}{ | c | c | c | c | c | }
    \hline
    \textbf{\texttt{username}} & \textbf{email} & \textbf{verified} & \textbf{\texttt{timestamp}} & \textbf{\texttt{address}} \\
    \hline
    \texttt{samp} & \texttt{samp9@}\ldots & \texttt{1} & \ldots\texttt{414} & \texttt{162}\ldots \\ \hline
    \texttt{karenk} & \texttt{karenk5@}\ldots & \texttt{1} & \ldots\texttt{337} & \texttt{171}\ldots \\ \hline
    \texttt{karenk} & \texttt{karenk5@}\ldots & \texttt{1} & \ldots\texttt{962} & \texttt{155}\ldots \\ \hline
    \texttt{karenk} & \texttt{karenk5@}\ldots & \texttt{1} & \ldots\texttt{720} & \texttt{171}\ldots \\ \hline
  \end{tabular}
\end{center}


\subsection{Transitive closure}
\label{sec:ra:tc}
%
One of the simplest common algorithms that may be implemented as a loop over fast relational algebra primitives, is computing the transitive closure of a relation or graph. Consider a relation $G \subseteq \mathbb{N}^2$ encoding a graph where each point $(a,b) \in G$ encodes the existence of an edge from node $a$ to node $b$.


\subsection{Datalog}
\label{sec:ra:tc}
%
Transitive closure is a simple example of deduction. At each...





\section{Hash-Tree Relational Algebra}
\label{sec:impl}
%
This section discusses our implementation of efficient distributed relational algebra. We employ a hybrid approach we call \emph{Hash-Tree} RA that consists of nesting B-tree key-value stores within a hash-table that can be partitioned across multiple cores or MPI nodes. Like the double-hashing approach 



\paragraph{Hybrid hash-table and b-tree} Our approach is to use an efficient key-value store, but to



\subsection{Hybrid Join}

...

\subsection{Distributed Join}
\begin{figure*}[h]
	\includegraphics[width=\textwidth]{results/join_new.pdf}
	\caption{Schematic Diagram to show different phases of hash-tree join.}
	\label{fig:join}
\end{figure*}
...


\subsection{Distributed Union}
\begin{figure*}[h]
	\includegraphics[width=\textwidth]{results/union_1.pdf}
	\caption{Schematic Diagram to show different phases of naive hash-tree union.}
	\label{fig:union_1}
\end{figure*}


We present two algorithms for distributed unions, buffered-hash-tree union and naive hash-tree union. As the name suggests buffered-hash-tree union buffers data across all sets that needs to be unioned before performing any communication or insertion related task. Buffered implementation has an extra memory overhead as opposed to the naive implementation where all graphs that needs to be unioned are processed one at a time.

While performing union of $n$ graphs, naive-hash-tree union involves $n$ epochs of communication and computation (one for every graph) as opposed to buffered-hash-tree union that uses buffering to limit the number of communication and computation epochs to one. 

\begin{figure}[h]
	\includegraphics[width=\columnwidth]{results/union_2.pdf}
	\caption{Schematic Diagram to show different phases of buffered hash-tree union.}
	\label{fig:union_2}
\end{figure}

With the naive approach, all processes iterate through the $n$ graphs one by one.
The input graphs can be read from files stored on the disk or can be read from the memory. If the graphs are red from disk, then first phase is that of parallel I/O, where processes access disjoint regions of the file to read equal number of tuples in parallel. Once the tuples are read, every process scans through the tuples and groups them into $nprocs$ (=$hashbuckets$) packets, ready to be sent across the network.
Target process (hash-bucket) of a tuple is computed based on the hash outcome of its key. For instance, target rank of a two column tuple $(a, b)$ would be $hash(a)\%nprocs$. We also perform preliminary deduplication to eliminate duplicate tuples in the input graph. The scan step is followed actual by all to all communication phase where tuples are sent to the appropriate processes (hash buckets). Once tuples arrive at a process, they are inserted into the relation container. This step performs the important task of deduplication of tuples across the graphs. 

With buffered-hash-tree union instead of processing the graphs one after the other, we read all the graphs at once, buffer the tuples and follow it with one cumulative step of hashing, communication and insertion. Both algorithms are presented in X.



\section{Evaluation}
\label{sec:eval}

%There are two goals of this sections 1) evaluate the performance of RA primitive operations Union and Join  2) evaluate the performance of transitive closure, which is a fixed point iteration algorithm.
The goal of this section is to evaluate the performance of parallel join, parallel union and parallel transitive closure at scale.
To this end, we individually study the computation and communication components of the RA operations.
Computation is dominated by insertion of tuples and the major challenge faced is that of deduplication.
We first study the efficacy of our btree-based relation container for inserts specifically in the context of deduplication.
All our RA operations involve an all to communication phase, hence, we perform a detailed MPI all to all benchmark.
Following these two experiments, we then benchmark the efficacy of parallel union, parallel join and parallel transitive closure over a wider range of graphs.
%We first study how our data structure performs, and how well are the all to all communication primitives are supported by super computers. More specifically, we study how fast can we insert into our btree based relation class, and how efficiently can it support the task of de-duplication. This is mostly studying the computation aspect of our algorithms. Next, with the all to all tests, we benchmark the communication aspect of our algorithms.


%Radix-hash join and merge-sort join are two of the most popularly used parallel implementations of the inner join operation. Both these algorithms involve partitioning the input data so that they can be efficiently distributed to the participating processes. For example, in the radix-hash approach a tuple is assigned to a process based on the hash output of the column-value on which the join operation is keyed. With this approach, tuples on both relations that share the same hash value are always assigned to the same process. For every tuple in the left-hand side of the join relation is matched against all the tuples of the right-hand side of the join relation. Fast lookup data-structures like hash tables, or radix-trees (TRIE) can be used to organize the tuples within every process. The initial distribution of data using hashing reduces the overall computation overhead by a factor of the number of processes (n).

%More recently (Barthels et al. 2015, 2017), there has been a concerted effort to implement JOIN operations on clusters using an MPI backend. The commonly used radix-hash join and merge-sort join have been re-designed for this purpose. Both these algorithms involve a hash-based partitioning of data so that they are be efficiently distributed to the participating processes and are designed such that inter-process communication is minimized. In both of these implementations one-sided communication is used for transferring data between process. With one-sided communication the initiator of a data transfer request can directly access parts of the remote memory and has full control where the data will be placed. Read and write operations are executed without any involvement of the target machine. This approach of data transfer involves minimal synchronization between particiapting processes and have been shown to scale better that traditional two-sided communication. The implementation of parallel join has shown promising performance numbers; for example, the parallel join algorithm of (Barthels et al. 2017) ran successfully at 4,096 processor cores with up to 4.8 terabytes of input data


\subsection{Dataset and HPC platforms}
\label{sec:datasets}
We have performed our experiments using open sourced graphs available at ~\cite{}.
The graphs represent a wide range in terms of the number of edges. The last three graphs are the largest available graphs.
Transitive closure of a graph with n edges can compute upto n^2 edges for a fully connected graph. The number of edges in the transitive closure of a graph depends on the connectivity of the input graph. We found our third graph with X edges to be highly connected; the transitive closure of the graph genertaed 260 billion edges, which is 3 terabytes in size.
-- wide range in terms of number of edges
-- The transitive closure of third graph generated 260 billion edges, which corresponds to 4 tera bytes of data.

\begin{table}[]
	\begin{tabular}{lllll}
		\begin{tabular}[c]{@{}l@{}}Input graph \\ edge count\end{tabular} & Union & Join & \begin{tabular}[c]{@{}l@{}}Transitive \\ Closure\end{tabular} & Graph name \\
		412148                                                            & \checkmark      & \checkmark     &                                                               &            \\
		2100225                                                           &       &      &                                                               &            \\
		6291408                                                           &       &      &                                                               &            \\
		59062957                                                          &       &      &                                                               &            \\
		136024430                                                         &       &      &                                                               &            \\
		180292586                                                         &       &      &                                                               &            \\
		240023949                                                         &       &      &                                                               &           
	\end{tabular}
\end{table}


The experiments presented in this work were performed on
Theta at the Argonne Leadership Computing Facility
(ALCF). Theta is a Cray XC30 with a peak
performance of X petaflops, 124, 608 compute cores, 332
TiB of RAM, and 7.5 PiB of online disk storage. We used
Edison Lustre file system (168 GiB/s, 24 I/O servers and 4
Object Storage Targets). 

%Default striping was used with the Lustre file system. Mira system contains 48 racks and 768K cores, and has a theoretical peak performance of 10 petaflops. Each node has 16 cores, with 16 GB of RAM per node. I/O and interprocessor communication travels on a 5D torus network. Every 128 compute nodes has two 2 GB/s bandwidth links to two different I/O nodes, making 4 GB/s bandwidth for I/O at most. I/O nodes are connected to file servers through QDR IB. Mira uses a GPFS file system with 24 PB of capacity and 240 GB/s bandwidth.


\begin{figure*}[t]
	{\includegraphics[width=.50\textwidth,  trim={0cm 0cm 0cm 0cm, 
			clip}]{results/inserts_with_duplicates.pdf}}\hfill%
	{\includegraphics[width=.50\textwidth,  trim={0cm 0cm 0cm 0cm,
			clip}]{results/inserts_with_no_duplicates.pdf}}\hfill%
	\centering
	\caption{Performance evaluation of relation class implemented with btree and unordered map. (left) All tuples are distinct, (right) There are four copies of every tuple being inserted. Relation implemented with btree out-performs the unordered-map implementation.}
	\label{fig:tuple_inserts}
\end{figure*}


\subsection{Relation}
\label{sec:relation}
-- Perform two benchmarks: insertion of tuples without any duplicate and insertion of tupes each with four duplicates.
-- Relation class implemented with btrees outperforms relation with hashes.
-- btree-relation scales with increase tuple count as opposed to hash based relation that fails to scale for large number of tuples.
-- For example hash based relation takes X seconds to insert Y tuples as opposed to only Z seconds taken by btree based relation.



\begin{figure*}[t]
	{\includegraphics[width=.50\textwidth,  trim={0cm 0cm 0cm 0cm, 
			clip}]{results/all_to_all_strong.pdf}}\hfill%
	{\includegraphics[width=.50\textwidth,  trim={0cm 0cm 0cm 0cm,
			clip}]{results/all_to_all_weak.pdf}}\hfill%
	\centering
	\caption{Strong (left) and Weak (right) scaling evaluation of MPI\_alltoallv function of MPI.}
	\label{fig:all_to_all}
\end{figure*}


\subsection{MPI\_All\_to\_Allv}
\label{sec:all_to_all}

-- perform both weak and strong scaling
-- configuration
-- weak scaling with accuracy X percent accuracy for per process load of Y versus Z percent accuracy for per process load of M.
-- Exhibit poor weak scaling for small load, as opposed to good weak scaling with larger volume of data.
-- X percent accuracy with strong scaling of X load. This corresponds to Y number of tuples.
-- Overall very good indication as all to all scales well, aggregate accuracy
-- different graphs would work better at different scales.

\subsection{Parallel Union}
\label{sec:union}
-- union of 7 graphs from table
-- union comprises of an io phase followed by communication and then followed by inserts
-- We compare two union types, one is where we perform io, comm and compute separates, the other is where we first perform io for all and then we bundle all our comm and then we have one phase of compute
-- scales well upto X cores., this is strong scaling.
-- faces work load deprecation at low core counts, needs more tuples for union to scale at high core counts
-- overall a good sign


\subsection{Parallel Join}
\label{sec:join}
-- setup, join of two of the largest graph, produces X number of tuples in output.
-- similar to union join scales well upto X cores, after that there is shortage of work and we do not see proper scaling.
-- communication stops to scale after X core counts. This is because there is lack of work.
-- 


\subsection{Transitive closure}
\label{sec:tc}

Computing the transitive closure of a graph involves repeated join operations until a fixed point is reached. We
use the previously discussed radix-hash join algorithm to distribute the tuples across all processes. The algorithm
can then be roughly divided into four phases: 1) Join 2) network communication 3) insertion 4) checking for a
fixed point. In our join phase every process concurrently computes the join output of the local tuples. In the next
phase every process sends the join output results to the relevant processes. This is a all-to-all communication
phase, which we implemet using MPI’s all to all routines. The next step involves inserting the join output result
received from the network to the output graph’s local partition. In the final step we check if the size of the
output graph changed on any process, if it does then we have not yet reached a fixed point and we continue to
another iteration of these 4 steps.
We performed a set of strong-scaling experiments to compute the transitive closure of graph with 412148
edges—the largest graph in the U. Florida Sparse Matrix set (Davis and Hu 2011). We used the Quartz supercomputer
at the Lawrence Livermore National Laboratory (LLNL). For our runs, we varied the number of processes
from 64 to 2048. A fixed point was attained after 2933 iterations, with the resulting graph containing 1676697415
edges. As can be seen in Figure 1, our approach takes 462 seconds at 64 cores and 235 seconds at 2048 cores, corresponds
to an overall efficiency of 6.25%. We investigated these timings further by plotting the timing breakdown
of by the four major components (join, network communication, join, fixed-point check) of the algorithm. We
observe (see Figure 2) that for all our runs the total time is dominated by computation rather than communication;
insert and join together tended to take up close to 90% of the total time. This is quite an encouraging result
as it shows that we are not bound primarily by the network bandwidth (at these scales and likely moderately
higher ones) and it gives us the opportunity to optimize the computation phase

\begin{figure*}[t]
	{\includegraphics[width=.50\textwidth,  trim={0cm 0cm 0cm 0cm, 
			clip}]{results/TC_1_6Billion.pdf}}\hfill%
	{\includegraphics[width=.50\textwidth,  trim={0cm 0cm 0cm 0cm,
			clip}]{results/TC_1_6Billion_breakdown.pdf}}\hfill%
	\centering
	\caption{Transitive closure of a graph with X edges.}
	\label{fig:tc_small}
\end{figure*}


\begin{figure*}[t]
	{\includegraphics[width=.50\textwidth,  trim={0cm 0cm 0cm 0cm, 
			clip}]{results/TC_260Billion.pdf}}\hfill%
	{\includegraphics[width=.50\textwidth,  trim={0cm 0cm 0cm 0cm,
			clip}]{results/TC_260Billion_breakdown.pdf}}\hfill%
	\centering
	\caption{Transitive closure of a graph with X edges.}
	\label{fig:tc_large}
\end{figure*}


\section{Related Work}
\label{sec:related}
%
Lorem ipsum dolar sit amat 





\section{Conclusion}

We have presented the first general algorithms for scalable relational algebra on supercomputers. Our approach addresses both representation and communication among portions of a distributed relation, laying the groundwork for scaling algorithms that require a pipeline of repeated operations on relations, or fixed-point iteration, such as logical and constraint problems, deductive databases, and static program analyses. We discovered that MPI's all-to-all communication paradigm is scalable, and suitable for relational algebra. Finally, we demonstrated the scalability of our operations, up to $65,\!536$ nodes, in the context of a fixed-point algorithm for computing transitive closure.

In future work, we plan to address distributing relations via hashing on multiple variables to improve performance for highly non-uniform relations. We also have plans to address combining task-level and data-level parallelism in the context of Datalog solvers---a problem that must naturally extend to large numbers of relations and fixed-point interation over equally large numbers of constraints.




%\begin{acks}
%To Robert, for the bagels and explaining CMYK and color spaces.
%\end{acks}

%

% The next two lines define the bibliography style to be used, and the bibliography file.
%\bibliographystyle{ACM-Reference-Format}
\bibliographystyle{abbrv}
\bibliography{sample-base}

% 
% If your work has an appendix, this is the place to put it.
%\appendix


\end{document}
