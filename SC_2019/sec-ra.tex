

\section{Relational Algebra}
\label{sec:ra}
%
Relational algebra provides a basis of operations on relations (i.e., predicates, or sets of tuples) sufficient to implement a broad range of algorithms for databases and queries, data analysis, machine learning, graph problems, and constraint logic problems \cite{}. Scaling these underlying primitives, and finding an effective strategy for parallel communication to distribute them across multiple nodes, is thus a avenue for scaling and distributing algorithms for high-performance program analyses, deductive databases, and other vital applications. This section reviews the most standard relational operations union, product, intersection, natural join, selection, renaming, and projection, along with their use in two related applications: graph problems and datalog solvers.

The Cartesian product of two finite enumerations $D_0$ and $D_1$ is defined $D_0 \times D_1 = \{ (d_0, d_1) \ |\ \forall d_0 \in D_0, d_1 \in D_1 \}$. A relation $R \subseteq D_0 \times D_1$ is some subset of this product that defines a set of \textit{related} pairs of elements drawn from the two domains. For example, if R were the relation $(\geq)$ over natural numbers, both domains $D_0$ and $D_1$ would be $\mathbb{N}$ and the relation could be defined $(\geq) = \{ (n_0, n_1) \ |\ n_0 \in \mathbb{N} \wedge n_1 \in \mathbb{N} \wedge n_0 \geq n_1 \}$.




\subsection{Transitive closure}
\label{sec:ra:tc}
%
...


\subsection{Datalog}
\label{sec:ra:tc}
%
Transitive closure is a simple example of deduction. At each...

